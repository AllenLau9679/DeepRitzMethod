\documentclass{article}

% if you need to pass options to natbib, use, e.g.:
% \PassOptionsToPackage{numbers, compress}{natbib}
% before loading nips_2018

\usepackage[preprint]{nips_2018}
% to compile a preprint version, e.g., for submission to arXiv, add
% add the [preprint] option:
% \usepackage[preprint]{nips_2018}

% to compile a camera-ready version, add the [final] option, e.g.:
% \usepackage[final]{nips_2018}

% to avoid loading the natbib package, add option nonatbib:
% \usepackage[nonatbib]{nips_2018}

\usepackage[utf8]{inputenc} % allow utf-8 input
\usepackage[T1]{fontenc}    % use 8-bit T1 fonts
\usepackage{hyperref}       % hyperlinks
\usepackage{url}            % simple URL typesetting
\usepackage{booktabs}       % professional-quality tables
\usepackage{amsfonts}       % blackboard math symbols
\usepackage{nicefrac}       % compact symbols for 1/2, etc.
\usepackage{microtype}      % microtypography
\usepackage{amsmath}
\usepackage{palatino}
\usepackage{tikz}
\usetikzlibrary{shapes.geometric, arrows}
\title{A Brief Report of Deep Ritz Method}

% The \author macro works with any number of authors. There are two
% commands used to separate the names and addresses of multiple
% authors: \And and \AND.
%
% Using \And between authors leaves it to LaTeX to determine where to
% break the lines. Using \AND forces a line break at that point. So,
% if LaTeX puts 3 of 4 authors names on the first line, and the last
% on the second line, try using \AND instead of \And before the third
% author name.

\author{
  Zeyu Jia\\School of Mathematical Science\\Peking University\\ \texttt{1600010603} \\
  \And
  Dinghuai Zhang\\School of Mathematical Science\\Peking University\\ \texttt{1600013525}\\
  \And
  Zhengming Zou\\School of Mathematical Science\\Peking University \\ \texttt{1600011089}
}

\begin{document}
% \nipsfinalcopy is no longer used

\maketitle

\begin{abstract}
\end{abstract}


\section{The Ritz Method}
\par The Ritz method is a method based on the principle of least action to find the approximation to eigenvalue equations that cannot be solved easily (or at all) analytically. Mathematically, the Ritz method can be used to approximate the solution of partial differential equations over a function space. When we want to seek a function $y(x)$that extremizes an integral $I(y(x))$. Suppose that we can approximate $y(x)$ by a linear combination of some linear independent  functions:
\begin{equation}
y(x)=\varphi_0(x)+k_1\varphi_1(x)+...+\varphi_N(x)
\end{equation}
Where $\varphi_{i}(x),i\in\{1,2,...,N\}$  are a set of basis of a function space that we are interested in. In many cases, we use a complete set of functions such as, polynomials or sines and cosines. And $k_i,i\in\{1,2,...,N\}$ are constants to be determined by variational and linear algebraic method. But if we don't just solve the problem in a limited function space, things can be different. 

\par Next we take the homogeneous Dirichlet boundary value problem of the Poisson equation as an example.
\begin{equation}
\left\{
\begin{aligned}
& \Delta u=-f(x),&x\in \Omega \\
 &u=0,&x\in \partial \Omega \\
 \end{aligned}
\right.
\end{equation}
The weak solution corresponding to the principle of least action is
\begin{equation}
\left\{
\begin{aligned}
 \text{find }\ u\in H(\Omega), s.t.\\
 I(u)=\min\limits_{v\in H(\Omega)}I(v)\\
  \end{aligned}
\right.
\end{equation}
Where $H$ is the set of admissible function. And using variational method, we can prove that 
\begin{equation} 
I(v)=\int_\Omega\left(\frac{1}{2}|\nabla v(x)|^2-f(x)v(x)\right)dx. 
\end{equation}
As a matter of fact, if we assume that v is an n-dimensional function, then
\begin{equation}\label{d_I}
\delta I(v)=\int _{\Omega} \left(\frac{\partial v}{\partial x_1}\cdot \delta \frac{\partial v}{\partial x_1}+...+\frac{\partial v}{\partial x_n}\cdot   \delta \frac{\partial v}{\partial x_n}-f(x)\delta      v(x)   \right)dx
\end{equation}
Using the integration by parts method for several items on the left,
\begin{equation}\label{int_by_parts}
\int _{\Omega}\frac{\partial v}{\partial x_1}\cdot \delta \frac{\partial v}{\partial x_1} dx= \frac{\partial v}{\partial x_1}\cdot \delta v\Big|_{\partial \Omega}-\int_{\Omega}\delta v \cdot \frac{\partial^2 v}{\partial x_1 ^2}dx=-\int_{\Omega}\delta v \cdot \frac{\partial^2 v}{\partial x_1 ^2}
\end{equation}
Substituting \eqref{int_by_parts} to \eqref{d_I}
\begin{equation}
\delta I(v)=-\int_{\Omega}\delta v(\Delta v+f(x))dx=0
\end{equation}

\section{Deep Ritz Method with neural networks}
\par Using the Ritz Method mentioned above, it is natural to think of solving pde with deep neural networks. Considering the PDE
\begin{equation}
\Delta u(x)+f(x)=0,\qquad x\in \Omega.
\end {equation}
\par According to Ritz Method, what we need to do is
\begin{equation}
\min\limits_{u\in H}{I(u)},
\end{equation}
where
\begin{equation}\label{I_equ}
I(u)=\int_\Omega\left(\frac{1}{2}|\nabla u(x)|^2-f(x)u(x)\right)dx,
\end{equation}
and $H$ is the set of admissible function. Our main idea is to facilitate the multi-layer neural network approximation function $u(x)$ and use the gradient descent algorithm to get the final result.\\

\subsection{Building trail function}
\par We mainly use a nonlinear transformation $x \to u_{\theta}(x)$ defined by deep neural networks to approximate function $u(x)$. Here $\theta$ denotes all parameters in our model. Similar to ResNet structure, we use several blocks to construct our networks, each block consists of two linear transformation, two activation functions and a residual connection. The $i$-th block can be expressed as 
\begin{equation}
s_i=\phi(W_{i2}\cdot\phi(W_{i1}\cdot s_{i_1}+b_{i1})+b_{i2})+s_{i-1}.\label{res_equ}
\end{equation}
\thispagestyle{empty}
% 流程图定义基本形状
	\begin{figure}
	\caption{Network Structure of Deep Ritz Method}
	\centering
	\tikzstyle{startstop} = [rectangle, rounded corners, minimum width = 2cm, minimum height=1cm,text centered, draw = black, fill=red]
	\tikzstyle{io} = [trapezium, trapezium left angle=70, trapezium right angle=110, minimum width=2cm, minimum height=1cm, text centered, draw=black, fill=purple]
	\tikzstyle{process} = [rectangle, minimum width=3cm, minimum height=0.5cm, text centered, draw=black, fill=pink]
	\tikzstyle{decision} = [diamond, aspect = 3, text centered, draw=black, fill=blue]
	% 箭头形式
	\tikzstyle{arrow} = [->,>=stealth]
	\begin{tikzpicture}[node distance=1cm]
	%定义流程图具体形状
	\node[io,  yshift = -1cm](in1){Input};
	\node[process, below of = in1, yshift = -0.5cm](fc1){FC layer(size m) + activation function};
	\node[process, below of = fc1, yshift = -0.5cm](fc2){FC layer(size m) + activation function};
	\node[decision, right  of = fc1, xshift = 5cm, yshift= -1cm](residual 1){Decision 1 ?};
	\node[process, below of = fc2 , yshift = -0.5cm](fc3){FC layer(size m) + activation function};
	\node[process, below of = fc3, yshift = -0.5cm](fc4){FC layer(size m) + activation function};
	\node[decision, right  of = fc3, xshift = 5cm, yshift= -1cm](residual 2){Decision 1 ?};
	\node[process, below of = fc4, yshift = -0.5cm](fc5){FC layer(size 1) };
	\node[io, below of = fc5, yshift = -0.5cm](out1){Output};
	\coordinate (point1) at (-3cm, -6cm);
	%连接具体形状
	\draw [arrow] (in1) -- (fc1);
	\draw [arrow] (in1) -| node  [right] {} (residual 1);
	\draw [arrow] (fc1) -- (fc2);
	\draw [arrow] (fc2) -- (fc3);
	\draw [arrow] (fc3) -- (fc4);
	\draw [arrow] (fc4) -- (fc5);
	%\draw (residual 1) -- node [above] {Y} (point1);
	\draw [arrow] (residual 1) |- node [right] {} (fc3);
	\draw [arrow] (fc3) -- (out1);
	\end{tikzpicture}
\end{figure}

where $s_{i}$ is the output of the $i$-th layer. $W_{i1},W_{i2}\in R^{m\times m},b_{i1},b_{i2}\in R^{m}$ are defined by linear transformation. $\phi$ is the activation function.

\par Because our pde involves the Laplace transform, we naturally hope that the second derivative of the function $u(x)$ is not a constant. So we need to pick a proper activation function to ensure the networks' nonlinearity. In our model, we have decided to use 
\begin{equation}
\phi(x)=\max\{x^3,0\} 
\end{equation}

\par The residual connection in (\ref{res_equ})
helps to avoid the gradient vanishing problems.
After several blocks, we use a linear transform to get the final result. The whole network can be expressed as
\begin{equation}
u_{\theta}(x)=a\cdot f_n(x) \circ ...\circ f_1(x)+b
\end{equation}
$f_i(x)$ is the $i$-th block. $a\in R^m,b\in R$ is defined by the final linear transform. Note that the input vector $x$ is not necessarily m-dimensional. In order to handle this mismatch, we can pad $x$ by a zero vector. In our model, we always assume $d<m$.
\par After building our trail function, we are left to minimize the $I(u)$ in (\ref{I_equ})

\subsection{Euler numerical integration method}
\par The first problem we need to solve is to calculate the integral in (\ref{I_equ})
 . For simplicity, define:
 \begin{equation}
 g(x,\theta)=|\nabla u(x)|^2-f(x)u(x)
 \end{equation}
 then the $I(u)$ can be expressed as
 \begin{equation}
 I(u)=\int _{\Omega}g(x,\theta)dx
 \end{equation}
 Obviously, it's impossible to calculate this integral directly. We use Euler numerical integration method to approximate the integral.
 \begin{equation}
 I(u)=L(x,\theta)=\frac{1}{N}\sum\limits_{i=1}^{N}g(x_i,\theta)
 \end{equation}
 Where each $x_i$ corresponds to a data point. Each data point is taken from the grid $[-1,1]\times [-1,1]$ in steps of 0.001.
 \subsection{The stochastic gradient descent algorithm}
 In deep learning, stochastic gradient descent (SGD) is a common method to minimize the loss function. In this problem, we also choose SGD method to minimize $I(u)$, which can be expressed as:
 \begin{equation}
 \theta^{k+1}=\theta^{k}-\eta \nabla_{\theta}\frac{1}{N}\sum\limits_{i=1}^{N}g(x_{i,k},\theta^k)
 \end{equation}
 where $k$ is the number of iterations.$\{x_{i,k}\}$ is the randomly selected data from the grid. In SGD, we only select a small number of data points from the grid at a time. Due to our limited computing capiticy, we should make a compromise  between computing the true gradient and the gradient at a single example. So we choose SGD to compute the gradient against more than one training example (also called a "mini-batch") at each step. In order to optimize our training process, we use the Adam version of SGD.


\section{Our improvements}
\par Along with the footsteps of professor E, we have achieved very good results. Based on the results already available, we want to make further improvements. 

\subsection{L2 regularization}

\subsection{Self adaptive}
\par In order to find out the tougher place and enhance the regularization, we use a self-adaptive regularization here. This idea is similar to adaptive finite element method, which refines the grid adaptively according to the difficulty of each grid. (this difficulty is described by the value of $f$ given the PDE $\Delta u = f$.) 
\par Here we adopt the self adaptive method on both boundary and interior. Suppose one time the neural network function is $f(x) = \mathrm{NN}(x, \theta_k)$, then we will use the following adaptive method to sample:
\begin{itemize}
	\item For the boundary restrictions, we sample points according to density function $|f(x)|$ where $x$ is on the boundary of $\partial\Omega$.
	\item For interior, we sample uniformly for the variation, and sample according to density function $|\Delta(x) - f(x)|$.
\end{itemize}
\par  According to this method, more points will be sampled at places where the neural network does not fit well, and less points will be sampled at places where the neural network fits well. 
\section{Numerical Results}

\subsection{The Poisson Equation}
\par Considering the Poisson equation:
\begin{equation}
\left\{
\begin{aligned}
 \Delta u=1,& x\in \Omega \\
 u=0, &x\in \partial \Omega \\
 \end{aligned}
\right.
\end{equation}

Here $\Omega =\{(x,y)| x^2+y^2<1\}$.
\par The exact solution to this problem is 
\begin{equation}
u=\frac{1}{4}(x^2+y^2-1)
\end{equation}

\par As described above, we use three blocks (six fully connected layers) and a final linear transform with $m=10$ to build our networks. There is a total of 671 parameters in our model. Considering the boundary condition, we need to make some modifications to our model. We have decided to use a penalty method and the modified function is:
\begin{equation}
I(u)=\int_{\Omega}\left(\frac{1}{2}|\nabla u(x)|^2-u(x)\right)dx+\gamma\int_{\Omega}|\Delta u(x) - 1|^2dx+\beta\int_{\partial \Omega}u(x)^2dx
\end{equation}
Where $\gamma\int_{\Omega}|\Delta u(x) - 1|^2dx$ is the regular term and we choose $\gamma=500$.

\section{Further discussion}

\section*{References}
\medskip

\small



\end{document}